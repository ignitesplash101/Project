\documentclass[11pt,letterpaper]{article}
\usepackage[margin=1in]{geometry}
\usepackage{graphicx}
\usepackage{hyperref}
\usepackage{cite}
\usepackage{amsmath}
\usepackage{enumitem}
\usepackage{tabularx,array,booktabs}
\usepackage{multirow}

\title{\textbf{Hedonic Forecast: ML Housing Prediction in Japan}}
\author{Team 90: Jianhuang Li, Shin Ying Chua, Wei Qi Thong, Ryan Kai Yan Seet}

\begin{document}
\date{}
\maketitle

\section*{Introduction\; }

We develop Hedonic Forecast, a forecasting pipeline that ingests 485{,}093 MLIT transactions (2005-Q3--2025-Q1) and produces ward- and 250\,m mesh forecasts for Tokyo's 23 wards and Sendai\cite{mlit_data}. The current models achieve ward-level MAE/RMSE of 18{,}595/29{,}651 JPY/m$^2$ and mesh-level MAE/RMSE of 21{,}317/121{,}977 JPY/m$^2$ while sustaining $R^2 \ge 0.95$, with predictions surfaced through an interactive Streamlit dashboard for real estate analysts, investors, and planners.
\vspace{1em}

\noindent Japan's official Japan Residential Property Price Index (JRPPI) is published monthly with roughly a three-month delay and only at coarse geographical definitions\cite{jrppi2020,jrppi_timelag}. Our work closes this gap by delivering neighbourhood-scale forecasts that can be refreshed with each quarterly MLIT data release. The pipeline contributes four implemented innovations: (1) quarterly mesh hedonic indices that retain transactions with missing building years through targeted indicators, (2) hierarchical forecasting in which mesh models ingest ward-level medians and lags to stabilize sparse geographies, (3) lightweight spatial encoding that maps MLIT coordinates to \emph{JIS X~0410} meshes without external geocoding\cite{stats_mesh,jshis_250m}, and (4) cold start smoothing using simple moving average and volatility features to dampen early-quarter spikes.

\section*{Problem Definition}

We address three tasks: (1) construct quarterly hedonic price indices at ward/mesh levels summarizing spatiotemporal dynamics from 2005--2025; (2) predict next-quarter median JPY/m$^2$ using only information available at prediction time; (3) provide interpretable, interactive exploration of spatial patterns. Inputs are MLIT arm's-length transactions (price, area, date, building year, location). Outputs include indices, one-step forecasts with uncertainty, standard metrics (MAE, RMSE, $R^2$), and future SHAP explainability\cite{shap2017}.

\vspace{-0.4em}
\section*{Background \& Literature Survey \;}

We build on the literature survey developed for the proposal and updated the evidence base where new results emerged. Spatial hedonic models capture location effects but often ignore temporal dynamics\cite{yoshida2024}, whereas temporal deep learners outperform classical baselines yet overlook neighbourhood context\cite{lstm2017,multivariate_lstm2022}. Transformer-style approaches promise to handle both of these issues but still incur heavy computation and interpretation costs, particularly for long sequences\cite{transform2024,transformers_survey}. Geostatistical studies for Tokyo emphasize multi-scale structure and rigorous cleaning pipelines\cite{spatiotemporal2023,hitotsubashi2023}, while mesh aggregation and spatial filtering highlight cross-ward spillovers that motivate our hierarchical design\cite{jsai2022,peng2022}. Finally, explainability research demonstrates how SHAP can be applied for a wide range of analytics models, including LSTMs\cite{shap2017,lundberg2020nmi,shap_lstm2025}. 

\section*{Data and Methods}

\subsection*{Dataset Overview}
We cleaned a total of 485{,}093 MLIT transactions (2005-Q3 to 2025-Q1) across Tokyo's 23 wards and Sendai's main wards, covering $\sim$3{,}000+ unique 250\,m meshes. Data cleaning addressed full-width numerals, Japanese era calendars, and missing building years (median-imputed by ward with \texttt{AgeUnknown} indicator). Spatial encoding derives \emph{JIS X~0410} 250\,m mesh codes directly from MLIT transaction coordinates (lat/lon), then collapses to district-quarter medians. A ward-centroid fallback is then applied if coordinates are missing. This avoids separate geocoding and reduces processing overheads substantially\cite{mlit_data,stats_mesh,jshis_250m}. Two panels are created as a result: ward$\times$quarter (28 wards) and mesh$\times$quarter, each with a set of median/dispersion price metrics, transaction counts, building statistics, and temporal keys.

\begin{table}[h!]
\centering
\caption{Dataset summary and temporal splits}
\label{tab:data_summary}
\begin{tabular}{@{}lr@{}}
\toprule
\textbf{Metric} & \textbf{Value} \\
\midrule
Total transactions & 485{,}093 \\
Time horizon & 2005-Q3 to 2025-Q1 \\
Unique 250m meshes & $\sim$3{,}000+ \\
Train/Val/Test (Ward) & 1{,}507 / 336 / 252 \\
Train/Val/Test (Mesh) & 26{,}818 / 6{,}337 / 4{,}762 \\
\bottomrule
\end{tabular}
\end{table}

\subsection*{Hedonic Price Index}
We estimate municipality-level indices using two-way fixed effects (PanelOLS):
\[
\ln(P_{it}) = \beta_0 + \beta_1 \ln(\text{Area}_{it}) + \beta_2 \text{Age}_{it} + \beta_3 \text{AgeUnknown}_{it} + \alpha_i + \gamma_t + \varepsilon_{it}
\]
where $\alpha_i$ (mesh FE) and $\gamma_t$ (quarter FE) capture spatial and temporal heterogeneity. The \texttt{AgeUnknown} dummy handles missing building year values without dropping observations. Quarterly granularity exceeds typical annual approaches. Fitted log prices are averaged by mesh-quarter, exponentiated, and normalized based on existing formulas\cite{c} to construct 2005--2025 indices.

\subsection*{Forecasting Models}
We train separate ward and mesh models with strict chronological splits: training through 2019-Q4, validation 2020-Q1 to 2021-Q4 (including COVID), testing 2022-Q1 to 2025-Q1. 

\noindent \textbf{Features:} Ward models (15 features) include 1- and 4-quarter price lags, quarter-over-quarter and year-over-year growth rates, 4-quarter moving averages, rolling standard deviations, seasonal indicators, time trend, transaction counts, building age/area, active mesh count, ward IDs. Mesh models (10 features) use price lags, growth rates, moving averages/standard deviations, counts, age/area, seasonality, trend. Early quarters with missing lags use zero-fill plus moving average/standard deviation features to handle historical data gaps for some municipalities.

\section*{Current Visualizations and Results}
\noindent The Streamlit dashboard currently runs locally and visualises housing data at the 250\,m mesh level. Users select the city, model, metric, and quarter to refresh the map, while tooltips expose predicted versus actual prices, average building age, floor area, and index values. Supplementary panels display leaderboard metrics for every model split, demonstrating the rubric-required progress on the visual component. Future work will replace placeholder SHAP graphics with computed explanations, add forecast overlays, and package the app for AWS deployment.

    \begin{figure}[h!]
        \centering
        \includegraphics[width=0.9\textwidth]{latex/viz_screenshot.PNG}
        \caption{Japan Housing Dashboard - Mesh-250m Interactive Map}
        \label{fig:myimage}
    \end{figure}

\noindent We report MAE, RMSE, and $R^2$ for one-step-ahead forecasts on the held-out 2022-Q1--2025-Q1 window, following standard accuracy guidance\cite{hyndman2006,tashman2000}.

\begin{table}[h!]
\centering
\caption{Test set performance (2022-Q1 to 2025-Q1)}
\label{tab:results}
\begin{tabular}{@{}lcccc@{}}
\toprule
\textbf{Level} & \textbf{Model} & \textbf{MAE} & \textbf{RMSE} & \textbf{$R^2$} \\
\midrule
\multirow{3}{*}{Ward} & Linear Regression & 18{,}595 & 29{,}651 & 0.995 \\
 & Random Forest & 44{,}251 & 112{,}745 & 0.933 \\
 & LightGBM & 47{,}911 & 116{,}949 & 0.928 \\
\midrule
\multirow{3}{*}{Mesh} & Random Forest & 21{,}317 & 121{,}977 & 0.953 \\
 & LightGBM & 26{,}228 & 133{,}619 & 0.943 \\
 & Linear Regression & 84{,}182 & 198{,}542 & 0.874 \\
\bottomrule
\end{tabular}
\end{table}
\noindent \textbf{Key findings:} Linear Regression remains the ward benchmark (MAE 18{,}595, $R^2$ 0.995), while Random Forest delivers the best mesh accuracy (MAE 21{,}317, $R^2$ 0.953) by capturing nonlinear interactions; LightGBM trails slightly at both levels.
\section*{Evaluation Plan and Challenges}
\noindent Building on the current Linear Regression and Random Forest baselines, we will execute the following two tasks, and perform additional finetuning of our models:
\begin{itemize}[leftmargin=*,labelsep=0.5em,noitemsep]
    \item \textbf{Baseline benchmarking:} Compare ward forecasts against lag-1 and four-quarter moving-average baselines and align them with JRPPI ward indices to quantify lift and lead/lag behaviour.
    \item \textbf{Scalability and Efficiency Evealuation} Analyze how model training and inference times scale with data volume by incrementally increasing the size of the training dataset (25\%, 50\%, 75\%, 100\% of transactions) and record wall-clock times for each configuration. These results helps to highlight trade-offs between model complexity and computational efficiency, informing future deployment feasibility.
\end{itemize}

\section*{Plan of activities\;}

\vspace{-0.4em}
\subsection*{Old Plan}
\begin{center}
\renewcommand{\arraystretch}{1.2}
\begin{tabularx}{\linewidth}{|>{\raggedright\arraybackslash}X|>{\raggedright\arraybackslash}X|l|c|c|}
\hline
\textbf{Activity} & \textbf{Checkpoints} & \textbf{Members} & \textbf{Start} & \textbf{Duration} \\
\hline
Literature survey & 12 peer-reviewed papers & All & Wk5 & 2 wks \\
Data cleaned \& features (ward/mesh) built & 90\% spatial tags & All & Wk7 & 2 wks \\
Baseline models (LR, RF, LightGBM) \& index validation & Midterm: hedonic indices built; baseline forecasts built and evaluated. & M1, M2 & Wk9 & 2 wks \\
Advanced models (LSTM) \& SHAP & SHAP plots generated for global and local & M3, M4 & Wk11 & 2 wks \\
Interactive Dashboard (map, leaderboard, SHAP) & Final: dashboard, report/video completed  & All & Wk13 & 2 wks \\
\hline
\end{tabularx}
\end{center}

\vspace{-0.4em}
\subsection*{New Plan}
\begin{center}
\renewcommand{\arraystretch}{1.2}
\begin{tabularx}{\linewidth}{|>{\raggedright\arraybackslash}X|>{\raggedright\arraybackslash}X|l|c|c|}
\hline
\textbf{Activity} & \textbf{Checkpoints} & \textbf{Members} & \textbf{Start} & \textbf{Duration} \\
\hline
Literature survey & 12 peer-reviewed papers completed & All & Wk5 & 2 wks \\
\hline
Data cleaning \& feature engineering (ward/mesh) & 100\% spatial tags completed; merged with transaction data & All & Wk7 & 3 wks \\
\hline
Housing price index construction & Hedonic price indices for Tokyo and Sendai wards completed & All & Wk9 & 2 wks \\
\hline
Baseline models & Baseline forecasts with LR, RF \& LightGBM completed & M1, M2 & Wk10 & 2 wks \\
\hline
Evaluation & Model performance compared across metrics - in progress (R\textsuperscript{2}, RMSE, MAE) & All & Wk13 & 1 wk \\
\hline
Advanced models (LSTM) \& SHAP & SHAP plots generated for global and local - in progress & M3, M4 & Wk12 & 2 wks \\
\hline
Interactive Dashboard (map, leaderboard, interpretability) & Dashboard integrated with first-cut results & All & Wk14 & 2 wks \\
\hline
\end{tabularx}
\end{center}

\noindent All team members contributed equally to all aspects of the project and report writing.

\newpage
\bibliographystyle{apalike}
\begin{thebibliography}{99}

\bibitem{mlit_data}
Ministry of Land, Infrastructure, Transport and Tourism (MLIT). (2005--present). \textit{Real Estate Information Library}. Retrieved from \url{https://www.reinfolib.mlit.go.jp/}

\bibitem{jrppi2020}
Ministry of Land, Infrastructure, Transport and Tourism, Real Estate and Construction Economy Bureau. (2020). \textit{Methodology of JRPPI: Japan Residential Property Price Index}. Retrieved from \url{https://www.mlit.go.jp/common/001360414.pdf}

\bibitem{jrppi_timelag}
Ministry of Land, Infrastructure, Transport and Tourism. (2015). \textit{Japan Residential Property Price Index and Residential Transaction Volume (August 2015)}. See p.~6: ``Time lag: About 3 months'' and coverage by geography. Retrieved from \url{https://www.mlit.go.jp/common/001110934.pdf}

\bibitem{yoshida2024}
Yoshida, T., \& Seya, H. (2024). Spatial prediction of apartment rent using regression-based and machine learning-based approaches with a large dataset. \textit{The Journal of Real Estate Finance and Economics}, \textit{69}(1), 1--28. \url{https://doi.org/10.1007/s11146-022-09929-6}

\bibitem{lstm2017}
Chen, X., Wei, L., \& Xu, J. (2017). House price prediction using LSTM. \textit{arXiv preprint arXiv:1709.08432}. \url{https://arxiv.org/abs/1709.08432}

\bibitem{c}
Haque, D. (2024). Transforming Japan real estate. \textit{arXiv preprint arXiv:2405.20715}. \url{https://arxiv.org/abs/2405.20715}

\bibitem{transformers_survey}
Wen, Q., Zhou, T., Zhang, C., Chen, W., Ma, Z., Yan, J., \& Sun, L. (2023). Transformers in time series: A survey. In \textit{Proceedings of IJCAI 2023} (Survey Track). \url{https://www.ijcai.org/proceedings/2023/0759.pdf}

\bibitem{spatiotemporal2023}
Muto, S., Sugasawa, S., \& Suzuki, M. (2023). Hedonic real estate price estimation with the spatiotemporal geostatistical model. \textit{Journal of Spatial Econometrics}. \url{https://link.springer.com/article/10.1007/s43071-023-00039-w}

\bibitem{jsai2022}
Mizuho Research \& Technologies, Ltd. (2022). \textit{機械学習を用いた土地価格の予測 [Prediction of Land Prices Using Machine Learning with Mesh-Based Neighbor Features ]}. JSAI Special Interest Group on Financial Informatics (SIG-FIN-029-61). Retrieved from \url{https://www.jstage.jst.go.jp/article/jsaisigtwo/2022/FIN-029/2022_61/_pdf}

\bibitem{stats_mesh}
Statistics Bureau of Japan. (n.d.). \textit{The Standard Grid Square and the Grid Square Code used for the Statistics}. Retrieved from \url{https://www.stat.go.jp/english/data/mesh/02.html}

\bibitem{jshis_250m}
National Research Institute for Earth Science and Disaster Resilience (NIED). (n.d.). \textit{What is the 250m-mesh code?} J-SHIS FAQ. Retrieved from \url{https://www.j-shis.bosai.go.jp/en/faq-250mmesh}

\bibitem{hitotsubashi2023}
Otsuki, K. (2023). \textit{A Study on Data-Cleansing Methods and Model-Update Algorithms for Real Estate Price Forecasting Models} [Doctoral dissertation, Hitotsubashi University]. Hitotsubashi University Repository. \url{https://hit-u.repo.nii.ac.jp/records/2048234}

\bibitem{shap2017}
Lundberg, S. M., \& Lee, S.-I. (2017). A unified approach to interpreting model predictions. In \textit{Advances in Neural Information Processing Systems 30} (pp. 4765--4774). \url{https://arxiv.org/abs/1705.07874}

\bibitem{lundberg2020nmi}
Lundberg, S. M., Erion, G., Chen, H., DeGrave, A., Prutkin, J. M., Nair, B., Katz, R., Himmelfarb, J., Bansal, N., \& Lee, S.-I. (2020). From local explanations to global understanding with explainable AI for trees. \textit{Nature Machine Intelligence}, \textit{2}(1), 56--67. \url{https://doi.org/10.1038/s42256-019-0138-9}

\bibitem{shap_lstm2025}
Sen, D., Deora, B. S., \& Vaishnav, A. (2025). Explainable deep learning for time series analysis: Integrating SHAP and LIME in LSTM-based models. \textit{Journal of Information Systems Engineering and Management}, \textit{10}(16s). \url{https://jisem-journal.com/index.php/journal/article/view/2627}

\bibitem{multivariate_lstm2022}
Kuber, V., Yadav, D., \& Yadav, A. K. (2022). Univariate and multivariate LSTM model for short-term stock market prediction. \textit{arXiv preprint arXiv:2205.06673}. \url{https://arxiv.org/abs/2205.06673}

\bibitem{hyndman2006}
Hyndman, R. J., \& Koehler, A. B. (2006). Another look at measures of forecast accuracy. \textit{International Journal of Forecasting}, \textit{22}(4), 679--688. \url{https://robjhyndman.com/papers/mase.pdf}

\bibitem{tashman2000}
Tashman, L. J. (2000). Out-of-sample tests of forecasting accuracy: An analysis and review. \textit{International Journal of Forecasting}, \textit{16}(4), 437--450. \url{https://www.sciencedirect.com/science/article/pii/S0169207000000650}

\bibitem{peng2022}
Peng, Z., \& Inoue, R. (2022). Identifying multiple scales of spatial heterogeneity in housing prices based on eigenvector spatial filtering approaches. \textit{ISPRS International Journal of Geo-Information}, \textit{11}(5), 283. \url{https://doi.org/10.3390/ijgi11050283}

\end{thebibliography}

\end{document}
